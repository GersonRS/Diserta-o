\chapter{Revisão da Literatura}
Definição dos principais conceitos e categorias a serem utilizadas na pesquisa. A definição da base teórica e conceitual é um momento fundamental, pois proporciona a sustentação da pesquisa científica. Este capítulo 2 como um todo deve começar numa página nova e costuma ter entre 5 e 10 páginas.
Aqui você deverá apresentar os principais conceitos associados ao tema que está sendo investigado. É interessante subdividir (2.1, 2.2, 2.n) de acordo com seus temas de pesquisa. Os subtópicos das referências conceituais também guardam relação com o problema de pesquisa. Você só deve falar de conceitos que tem relação com seu problema. Aqui você deve mostrar a contribuição dos diversos autores que falam sobre o assunto que você está investigando, então é necessário citá-los. 
Muito cuidado neste momento para não incorrer em plágio, que é a apropriação indevida de textos produzidos por outros autores sem a devida citação conforme norma. Outro cuidado é para que o texto não fique parecendo uma colcha de retalhos, com as diversas partes sem nenhuma conexão. Procure encadear bem o texto.